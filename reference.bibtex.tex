

@article{Fader2005,
author = {Fader, Peter S. and Hardie, Bruce G. S. and Lee, Ka Lok},
title = {“Counting Your Customers” the Easy Way: An Alternative to the Pareto/NBD Model},
journal = {Marketing Science},
volume = {24},
number = {2},
pages = {275-284},
year = {2005},
doi = {10.1287/mksc.1040.0098},

URL = { 
        https://doi.org/10.1287/mksc.1040.0098
    
},
eprint = { 
        https://doi.org/10.1287/mksc.1040.0098
    
}
,
    abstract = { Today’s managers are very interested in predicting the future purchasing patterns of their customers, which can then serve as an input into “lifetime value” calculations. Among the models that provide such capabilities, the Pareto/NBD “counting your customers” framework proposed by Schmittlein et al. (1987) is highly regarded. However, despite the respect it has earned, it has proven to be a difficult model to implement, particularly because of computational challenges associated with parameter estimation. We develop a new model, the beta-geometric/NBD (BG/NBD), which represents a slight variation in the behavioral “story” associated with the Pareto/NBD but is vastly easier to implement. We show, for instance, how its parameters can be obtained quite easily in Microsoft Excel. The two models yield very similar results in a wide variety of purchasing environments, leading us to suggest that the BG/NBD could be viewed as an attractive alternative to the Pareto/NBD in most applications. }
}


@article{Reinartz2000,
 ISSN = {00222429},
 URL = {http://www.jstor.org/stable/3203475},
 abstract = {Relationship marketing emphasizes the need for maintaining long-term customer relationships. It is beneficial, in general, to serve customers over a longer time, especially in a contractual relationship. However, it is not clear whether some of the findings observed in a contractual setting hold good in noncontractual scenarios: relationships between a seller and a buyer that are not governed by a contract or membership. The authors offer four different propositions in this study and subsequently test each one in a noncontractual context. The four propositions relate to whether (1) there exists a strong positive customer lifetime-profitability relationship, (2) profits increase over time, (3) the costs of serving long-life customers are less, and (4) long-life customers pay higher prices. The authors develop arguments both for and against each of the propositions. The data for this study, obtained from a large catalog retailer, cover a three-year window and are recorded on a daily basis. The empirical findings observed in this study challenge all the expectations derived from the literature. Long-life customers are not necessarily profitable customers. The authors develop plausible explanations for findings that go against the available evidence in the literature and identify three indicators through discriminant analysis that can help managers focus their efforts on more profitable customers. The authors draw several marketing implications and acknowledge the limitations of the study.},
 author = {Werner J. Reinartz and V. Kumar},
 journal = {Journal of Marketing},
 number = {4},
 pages = {17--35},
 publisher = {American Marketing Association},
 title = {On the Profitability of Long-Life Customers in a Noncontractual Setting: An Empirical Investigation and Implications for Marketing},
 volume = {64},
 year = {2000}
}


@article{DRÈZE 2011,
 ISSN = {00222437},
 URL = {http://www.jstor.org/stable/23033430},
 abstract = {The authors examine the impact of successfully attaining a goal on future effort directed at attaining the same goal. Using data from a major frequent-flier program, they demonstrate empirically how success contributes to an increase in effort exhibited in consecutive attempts to reach a goal. They replicate the effects in a laboratory study that shows that the impact of success is significant only when the goal is challenging. They also show how progress enhances perceptions of self-efficacy and how successfully completing the task provides an added boost, supporting the notion that self-learning is the principle mechanism driving their results.},
 author = {XAVIER DRÈZE and JOSEPH C. NUNES},
 journal = {Journal of Marketing Research},
 number = {2},
 pages = {268--281},
 publisher = {American Marketing Association},
 title = {Recurring Goals and Learning: The Impact of Successful Reward Attainment on Purchase Behavior},
 volume = {48},
 year = {2011}
}


@article{Iyengar2011,
author = {Iyengar, Raghuram and Jedidi, Kamel and Essegaier, Skander and Danaher, Peter J.},
title = {The Impact of Tariff Structure on Customer Retention, Usage, and Profitability of Access Services},
journal = {Marketing Science},
volume = {30},
number = {5},
pages = {820-836},
year = {2011},
doi = {10.1287/mksc.1110.0655},

URL = { 
        https://doi.org/10.1287/mksc.1110.0655
    
},
eprint = { 
        https://doi.org/10.1287/mksc.1110.0655
    
}
,
    abstract = { Past research in marketing and psychology suggests that pricing structure may influence consumers' perception of value. In the context of two commonly used pricing schemes, pay-per-use and two-part tariff, we evaluate the impact of pricing structure on consumer preferences for access services. To this end, we develop a utility-based model of consumer retention and usage of a new service. A notable feature of the model is its ability to capture the pricing structure effect and measure its impact on consumer retention, usage, and pricing policy. Using data from a pricing field experiment for a new telecommunication service, we find that consumers derive lower utility from consumption under a two-part tariff than pay-per-use pricing, resulting in lower retention of customers and lower usage of the service. Specifically, our demand analysis shows that a two-part tariff structure leads to an average decline of 10.5\% in the annual retention rate and an average decrease of 38.7\% in yearly usage relative to pay-per-use pricing after controlling for income effects. Despite the higher customer churn and lower usage, we find that the two-part tariff is still the profit-maximizing pricing structure. However, our results show that if firms ignore the pricing structure (or access fee) effect, then they would overcharge customers for the access fee and undercharge them for the per-minute price. Translated in terms of profitability, the failure to account for the access fee effect leads to a reduction of 11\% in firm profit. }
}


@article{Ascarza2012,
author = {Eva Ascarza and Anja Lambrecht and Naufel Vilcassim},
title ={When Talk is “Free”: The Effect of Tariff Structure on Usage under Two- and Three-Part Tariffs},
journal = {Journal of Marketing Research},
volume = {49},
number = {6},
pages = {882-899},
year = {2012},
doi = {10.1509/jmr.10.0444},

URL = { 
        https://doi.org/10.1509/jmr.10.0444
    
},
eprint = { 
        https://doi.org/10.1509/jmr.10.0444
    
}
,
    abstract = { In many service industries, firms introduce three-part tariffs to replace or complement existing two-part tariffs. In contrast with two-part tariffs, three-part tariffs offer allowances, or “free” units of the service. Behavioral research suggests that the attributes of a pricing plan may affect behavior beyond their direct cost implications. Evidence suggests that customers value free units above and beyond what might be expected from the change in their budget constraint. Nonlinear pricing research, however, has not considered such an effect. The authors examine a market in which three-part tariffs were introduced for the first time. They analyze tariff choice and usage behavior for customers who switch from two-part to three-part tariffs. The findings show that switchers significantly “overuse” in comparison with their prior two-part tariff usage. That is, they attain a level of consumption that cannot be explained by a shift in the budget constraint. The authors estimate a discrete/continuous model of tariff choice and usage that accounts for the valuation of free units. The results show that the majority of three-part-tariff users value minutes under a three-part tariff more than they do under a two-part tariff. The authors derive recommendations for how the provider can exploit these insights to further increase revenues. }
}


